\documentclass[10pt,conference,compsocconf]{IEEEtran}

%\usepackage{times}
%\usepackage{balance}
\usepackage{url}
\usepackage{graphicx}	% For figure environment


\begin{document}
\title{Inpainting by Smoothing}

\author{
  Jan Deriu, Rolf Jagerman, Kai-En Tsay\\
  Department of Computer Science, ETH Zurich, Switzerland
}

\maketitle

\begin{abstract}
  This guide should be a
  starting point for further development of writing skills.
\end{abstract}

\section{Introduction}

The aim of writing a paper is to infect the mind of your reader with
the brilliance of your idea~\cite{jones08}. 

\section{Smoothing algorithm}
\label{sec:structure-paper}

Scientific papers usually begin with the description of the problem,
justifying why the problem is interesting. Most importantly, it argues
that the problem is still unsolved, or that the current solutions are
unsatisfactory. This leads to the main gist of the paper, which is
``the idea''. The authors then show evidence, using derivations or
experiments, that the idea works. Since science does not occur in a
vacuum, a proper comparison to the current state of the art is often
part of the results. Following these ideas, papers usually have the
following structure:


\section{Results}

Scientific papers usually begin with the description of the problem,
justifying why the problem is interesting. Most importantly, it argues
that the problem is still unsolved, or that the current solutions are
unsatisfactory. This leads to the main gist of the paper, which is
``the idea''. The authors then show evidence, using derivations or
experiments, that the idea works. Since science does not occur in a
vacuum, a proper comparison to the current state of the art is often
part of the results. Following these ideas, papers usually have the
following structure:

\section{Discussion}

Scientific papers usually begin with the description of the problem,
justifying why the problem is interesting. Most importantly, it argues
that the problem is still unsolved, or that the current solutions are
unsatisfactory. This leads to the main gist of the paper, which is
``the idea''. The authors then show evidence, using derivations or
experiments, that the idea works. Since science does not occur in a
vacuum, a proper comparison to the current state of the art is often
part of the results. Following these ideas, papers usually have the
following structure:

\section{Future work}
Scientific papers usually begin with the description of the problem,
justifying why the problem is interesting. Most importantly, it argues
that the problem is still unsolved, or that the current solutions are
unsatisfactory. This leads to the main gist of the paper, which is
``the idea''. The authors then show evidence, using derivations or
experiments, that the idea works. Since science does not occur in a
vacuum, a proper comparison to the current state of the art is often
part of the results. Following these ideas, papers usually have the
following structure:

\section{Summary}


\bibliographystyle{IEEEtran}
\bibliography{cil_report}
\end{document}
