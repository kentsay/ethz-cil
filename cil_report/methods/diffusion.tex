\subsection{Regular diffusion}
The idea of diffusion is to fix the known regions of the image and let them diffuse into the unknown regions. This can be done very efficiently by using a convolutional kernel. By iteratively convolving a kernel with the entire image and then restoring the known pixels we can obtain an inpainted image. This process is described in algorithm \ref{alg:diffusion}. The quality of the solution heavily depends on the kernel used. A variety of kernels are used which are displayed in figure \ref{fig:diffusion}.

\begin{algorithm}
	\KwIn{Image $I$, mask $M$, kernel $K$ and threshold $\epsilon$}
	\KwResult{Reconstructed image $I_{r}$}
	$I_{prev} = 0_{size(I)}$\;
	$I_{r} = I$\;
	\While{$\|I_{r} - I_{prev} \|_{F} > \epsilon$}{
		$I_{r} = \text{convolve}(I_{r}, K)$\;
		$I_{r} = I_{r} \circ \mathrm{I}_{M = 0} + I \circ \mathrm{I}_{M \neq 0}$ \;
	}
	\quad
\caption{Diffusion algorithm for inpainting. Because the kernel $K$ will need to refer to pixels outside of the image, we replicate the outer borders of $I$ outward.}
\label{alg:diffusion}
\end{algorithm}